\chapter{Philosophical motivations}

It took decades before Peirce's generalization of the matrix method to three values was even discovered. The first mention of this development seems to have been in an article Turquette produced for an edited volume entitled \textit{Studies in the Philosophy of Charles Sanders Peirce} \citep{turquette1964studies}. To date, the only attempts there seem to have been to exposit Peirce's philosophical motivations are due to Fisch and Turquette \citeyearpar{fisch_peirces_1966}, and Lane \citeyearpar{lane_peirces_1999}. After Fisch and Turquette published that first article together, Turquette went on to develop various formal aspects of Peirce's triadic logic, mostly ignoring the philosophical side of the equation.

In this section, I will give an account of each of these attempts to explain why Peirce saw the need to deviate from the classical two valued logic he helped to create, and weigh their claims against the evidence that can be gleaned from Peirce's notebook. Fisch and Turquette lean towards his notions of triadic modality as an explanation, however they offer a couple of other possibilities along the way. Lane on the other hand, insists that modality had nothing to do with the matter, and instead draws his explanation based on Peirce's views on continuity and continua. I will first give an account of Fisch and Turquette's views and then turn to Lane.

\section{Max Fisch and Atwell Turquette's account}

Fisch and Turquette begin by discussing the various operators and formal developments on the first three pages of the notebook I have discussed. They note that on the formal side, Peirce seems to have been motivated by a concern for duality and functional completeness when he defined his operators as demonstrated by the fact that every one of them has its $\bar{x}$ dual defined also. But when it comes to his philosophical reasons, Fisch and Turquette are much more tentative. They offer roughly three possibilities: 1) that Peirce was motivated by considerations of modality, as Łukasiewicz was, 2) that his motivations were due to his doctrine of ``Tychism,'' which basically holds that there is fundamental indeterminacy in the world, and 3) that his triadic logic was connected to the ``three dimensional logic'' of Hugh MacColl, who was a long time correspondent of Peirce.

They begin their explanation stating that ``It is clearly indicated that the motivation arises from problems associated with the kind of proposition which `has a lower mode of being such that it can neither be determinately $P$, nor determinately not-$P$' --- assuming that the proposition in question is of the form $S$ is $P$'' \citep[77]{fisch_peirces_1966}. They draw this evidence from \href{https://iiif.lib.harvard.edu/manifests/view/drs:15255301$645i}{seq. 645}, where Peirce states as much himself, and claim that this is enough to suggest that he was motivated by concerns for modality, as Łukasiewicz was. More specifically, they claim he was interested in modal issues in which a third truth value seems necessary to evaluate propositions, such as future contingents. The most famous example is Aristotle's future sea battle case. If I pronounce `there will be a sea battle tomorrow', at the time of my utterance it is unclear what the truth value of the proposition is. Whether or not there does happen to be a sea battle the next day, it may still seem inappropriate to say the proposition is true or false, and we may want to reserve an intermediate value for the proposition. Nonetheless, Peirce's one remark on \href{https://iiif.lib.harvard.edu/manifests/view/drs:15255301$645i}{seq. 645} seems rather thin evidence to justify the claim that it was these kinds of modal propositions he was hoping to capture.

To further support their claim, Fisch and Turquette draw our attention an entry in \textit{The Prescott Book} (MS 277) from January 1908. The passage discusses modality in connection with ``potentiality'', ``actuality'', and ``necessitation'' \citep{fisch_peirces_1966}. It reads: \begin{quotation}
\noindent\textit{Potentiality} is the absence of Determination (in the usual broad sense) not of a mere negative kind but a positive capacity to be Yea or Nay; not ignorance but a state of being...\\ \textit{Actuality} is the Act which determines the merely possible...\\ \textit{Necessitation} is the support of Actuality by reason...
\end{quotation}
\noindent They then go on to cite the passage concerning the three universes in the Logic Notebook from August 1908 (\href{https://iiif.lib.harvard.edu/manifests/view/drs:15255301$550i}{seq. 550}). This provides evidence that Peirce was thinking about modality throughout the year leading up to his experiments in triadic logic. It also provides a possible explanation of what Peirce meant when he writes of lower modes of being on \href{https://iiif.lib.harvard.edu/manifests/view/drs:15255301$645i}{seq. 645} (\ref{fig:645}). His comment seems to imply a commitment to a hierarchy of modes of being, and it certainly makes sense to think of ``potentiality'' being at the bottom and ``necessitation'' on the top if he really was concerned with modality. Furthermore, if I am correct that there is a connection between the ``special universes'' in the title of \href{https://iiif.lib.harvard.edu/manifests/view/drs:15255301$639i}{seq. 639} (\ref{fig:639r}) and the ``three universes'' Peirce discusses in the notebook previously, then this would be strong evidence in favor of Fisch and Turquette's view that he was operating with some notion of modality in mind.

Peirce continued thinking and writing about modality in triadic terms throughout the final years of his life and it seems fairly clear that when he did so he had this kind of hierarchy in mind. In an unpublished essay cited by Fisch and Turquette, entitled \textit{The Art of Reasoning Elucidated}, he writes:\begin{quotation} \noindent``Now, in this respect, a simply assertory proposition differs just half as much from the assertion of a Possibility, or that of a Necessity, as these two differ from each other. For, as we have seen above, that which characterizes and defines an assertion of Possibility is its emancipation from the Principle of Contradiction, while it remains subject to the Principle of Excluded Third; while that which characterizes and defines an assertion of Necessity is that it remains subject to the Principle of Contradiction, but throws off the yoke of the Principle of Excluded Third; and what characterizes and defines an assertion of Actuality, or simple Existence, is that it acknowledges allegiance to both formulae, and is thus just midway between the two rational ``Modals'', as the modified forms are called by all the old logicians.'' (MS 678, 1910)\end{quotation}
\noindent Notice that when Peirce writes of ``Actuality'' he locates it between the possible and the necessary, suggesting a hierarchical ordering. Some of what is contained in this passage is perfectly consistent with the way we currently think of modal propositions too. While we would not normally say that the law of non-contradiction does not apply to modals, it is trivial to say that for any possible proposition $P$, `it is possible that $P$ and it is possible that not $P$.' This is likely what Peirce means when he claims potentials are emancipated from this principle. We can say something similar with regard to necessities and the principle of excluded middle. While we would not say that PEM fails for necessary statements, we also would not say of any proposition $P$ that `it is necessary that $P$ or it is necessary that not $P$' since it could be the case that neither. This is likely what Peirce meant when discussing these principles, although admittedly, this is not the standard way of understanding either of them.

Fisch and Turquette also use these notions to clarify what Peirce says on \href{https://iiif.lib.harvard.edu/manifests/view/drs:15255301$645i}{seq. 645} (\ref{fig:645}). They claim ``Essentially, Peirce seems to be saying that triadic logic may be interpreted as a modal logic which is designed to deal with the indeterminacies resulting from that mode of being which Peirce has called `Potentiality' and `Real Possibility''' \citep{fisch_peirces_1966}. This is why on that page Peirce claims that dyadic logic is ``not universally false it is only $L$.'' He may be saying that dyadic logic is limited because it fails to account for these real indeterminacies.

This brings us to their second point about Peirce's possible motivations: his Tychism. Unlike other philosophers working on logic around the time these pages were written, Peirce was committed to the view that there is fundamental indeterminacy in the world that cannot be removed by rendering propositions less ambiguous.\footnote{Unlike Bertrand Russell who in 1906 seems to have held that indeterminacy can always be removed by carrying a propositions determination further. So, for example, the indeterminacy of the proposition `Mrs. Brown is at home' can be removed by making the proposition more specific, i.e. by specifying a time and date, as in `Mrs. Brown was at home on the afternoon of \today.' \citep{russell_review_1906}} For Peirce, this indeterminacy is metaphysical in nature, and does not necessarily result merely from a lack of knowledge or the language we use. This follows from his 1898 definition of Tychism, as ``the doctrine that absolute chance is a factor in the universe'' (CP 6.201). Contrary to some authors of his time, like Russell, Peirce did not believe that ``the universe is [necessarily] regulated by law down to every detail" (Ibid). He was not a determinist, and thought that some facts of the world might come down to arbitrary chance. A little less than two weeks after he completed \href{https://iiif.lib.harvard.edu/manifests/view/drs:15255301$645i}{seq. 645} (\ref{fig:645}), Peirce writes in a letter to William James ``I hold to my `tychism' more than ever.' Fisch and Turquette use this evidence to connect Peirce to others who have held a similar belief in some kind of irreducible indeterminacy, like Hans Reichenbach or Werner Heisenberg, both of whom were motivated by undecidable statements about quantum mechanics. They further remark that Gödel's undecidable statements in mathematics could be another example of this kind of indeterminacy. Obviously Peirce could not have been aware of any of these developments, however his tychism might be seen as anticipating these kinds of results.

So it seems that Peirce's introduction of the value `$L$' to logic may have been a way of importing his tychism to logic. This would be consistent with complaints Peirce made about the ``oldfashioned logicians''\footnote{Presumably, these are more conservative logicians who were resistant to the advances made in logic by Boole, De Morgan, and Peirce, in favor of traditional syllogistic logic.} in a draft of the just mentioned letter, written just three days after seq. 645 (\ref{fig:645}). According to Fisch and Turquette's reproduction, it reads:
\begin{quotation} 
\noindent``I have long felt that it is a serious defect in existing logic that it takes no heed of the \textit{limit} between two realms. I do not say that the Principle of Excluded Middle is downright \textit{false}; but I \textit{do} say that in every field of thought whatsoever there is an intermediate ground between \textit{positive assertion} and \textit{positive negation} which is just as Real as they. Mathematicians always recognize this, and seek for that limit as the presumable lair of powerful concepts; while metaphysicians and oldfashioned logicians, --- the sheep [and] goat seperators, --- never recognize this. The recognition does not involve any denial of existing logic, but it involves a great addition to it.''
\end{quotation}
\noindent Is Peirce making reference to his triadic logic here? His remarks about this ``Real intermediate ground'' may be driven by his commitment to his tychism. Furthermore, his remarks here that acknowledging such a middle ground ``does not involve any denial of existing logic, but it involves a great addition to it'' seem awfully similar to his remarks about the difference between dyadic and triadic logic on \href{https://iiif.lib.harvard.edu/manifests/view/drs:15255301$645i}{seq. 645} (\ref{fig:645}). Thus, it seems Peirce's three-valued efforts might be an attempt to insert his tychism into logic. This explanation of Peirce's motivation is not necessarily distinct from the possibility that he was motivated by modal considerations, but it may help elucidate his understanding of modality.

The final point that Fisch and Turquette investigate has to do with the extent to which Peirce's triadic logic was related to the ``three dimensional'' logic of his long time correspondent, Hugh MacColl. MacColl's approach to logic involves a division of statements into three kinds: certain, impossible, and variable \citep{maccoll_symbolic_1906}. His logic is to a certain extent three-valued in that it accounts for the kinds of propositions he calls ``variable,'' which he sometimes refers to as `possible.' However, it is unclear whether he intended these variable propositions to take on a intermediate truth value, or if they were of unknown value and his logic contained truth gaps. This difficulty has led some to claim that MacColl is not actually a pioneering figure in the many-valued turn in logic \citep{conjunction_peter_1998}. However, it could be argued that any logic that admits truth gaps might just as easily be interpreted as a three-valued system, where the gaps are interpreted as the additional value.

Questions of his status in the history of non-classical logic aside, there is strong evidence that Peirce was aware of this particular aspect of MacColl's work. The two had been long time correspondents and admirers of each others work. In 1906, Peirce wrote a draft of a letter to MacColl inquiring about his new book:
\begin{quotation}
``P.O. Millford Pa 1906 Nov 16

My dear Sir:

Although my studies in symbolic logic have differed from yours in that my own aim has not been to apply the system to the working out of problems, as yours has, but to aid in the study of logic itself, nevertheless I have always thought that you alone, so far as I know, except myself, have understood how the matter ought to be treated by making the elements propositions on predicates and not common nouns. I beg have to send you here with a paper setting forth, in outline only, my system of existential graphs, which exhibits the logic of relations in the simplest possible manner.

I see by ``Nature'' of Nov 1 that you have a new book on the subject. My circumstances are so reduced that I can no longer purchase books. I notice them, however, for three or four important journals in this country, and should like very much to hear...'' (L 261).\footnote{Letter 261 in \textit{The Robin Catalogue}.}
\end{quotation}
\noindent The draft does not appear to ever have been completed and it is unclear whether he wrote to MacColl this late in his life. However, the draft does indicate that Peirce was aware of MacColl's later work. The article in ``Nature'' that he refers to here is a book review covering four volumes on logic released that year, among them MacColl's \textit{Symbolic Logic and its applications}. It makes explicit reference to MacColl's inclusion of variable propositions and notes a couple of criticisms against the feature. Thus, it is also likely that Peirce was aware of the aspect of MacColl's logic that bears a connection to his own triadic logic. Perhaps the reason for Peirce's interest in MacColl's book was that he thought it might offer some help with the issues concerning triadic modality that so consumed him around this time. However, Peirce never explicitly mentions MacColl or his book elsewhere so it is unlikely the connection goes any further than this.

Fisch and Turquette seem to make a strong case that in experimenting with three-valued logic, Peirce was motivated by concerns for his doctrine of tychism, which more broadly factors into his understanding of modality. At this point it seems likely that Peirce, like Łukasiewicz, thought it necessary to deviate from classical logic in order to deal with certain kinds of modal propositions. This connection is strengthened by considering some of what is written on the pages Fisch and Turquette neglect to mention. However, this suggestion runs contrary to the view Lane takes on Peirce's experiments. There is one major defect with Fisch and Turquette's account and this is that they pay no special attention to Peirce's ink blot example on \href{https://iiif.lib.harvard.edu/manifests/view/drs:15255301$645i}{seq. 645} (\ref{fig:645}). The only mention of it is in a sentence that declares it ``is not very helpful in providing an answer'' \citep{fisch_peirces_1966}. Lane's account explores this example in much greater detail.

\section{Robert Lane's account}

Writing more than 30 years after Fisch and Turquette, Robert Lane expresses an explicitly very different view of what Peirce was up to with his triadic logic. On Lane's view, Peirce was motivated not by concern for modality, but by his understanding of continuity and the continuum. In this section I will elaborate on Lane's account of Peirce's motivations. I will begin by explaining why Lane thinks we should reject the view that locates Peirce's motivation in the realm of modality. I will then elaborate on Lane's positive claim, that Peirce was instead motivated by issues resulting from continuity.

Lane's route to his denial that Peirce was not here concerned with triadic modality is somewhat long. It begins with an exposition of Peirce's views on the principle of excluded middle (PEM) on the one hand, and the principle of non-contradiction (PC) on the other. His account here is an expansion of a previous paper on PEM and PC \citep{lane_peirces_1999}.

A crucial distinction involved with Peirce's understanding of these principles, according to Lane, ``is the distinction between saying of a logical principle, on the one hand, \textit{that it does not apply to a proposition}, and on the other, \textit{that it is false with regard to that proposition}" \citep{lane_peirces_1999}. According to this distinction, Lane claims that Peirce's $L$-propositions are ones that PEM still \textit{applies} to, but is nonetheless false in regards to. There are some obvious difficulties with this distinction, and it may involve a less than standard view of what it is to be a logical principle. It is not at all clear what sense there is in saying that a logical principle applies to a proposition that falsifies it. At face value, it seems a proposition that makes PEM false is precisely a proposition that PEM does not apply to. Lane indicates that Peirce made this distinction only once in a manuscript written in the same year he created his triadic logic, but he does not indicate exactly where this page can be found. Nevertheless, these remarks are somewhat consistent with Peirce's claim on \href{https://iiif.lib.harvard.edu/manifests/view/drs:15255301$645i}{seq. 645} (\ref{fig:645}) that triadic logic ``does not reject entirely the principle of excluded middle.''

While Peirce may have only made this particular distinction once, there are plenty of examples of him writing about propositions that PEM and PC do not apply to (though he doesn't say they are false in regard to these propositions). These are always general propositions and vague propositions respectively. As an example of this, Lane draws our attention to the following passage: ``anything is \textit{general} insofar as the principle of excluded middle does not apply to it and is \textit{vague} insofar as the principle of contradiction does not apply to it. '' (CP 5.448, \textit{Issues in Pragmaticism}, 1905). Notice the similarity between Peirce's remark here and his remarks about potentials and necessities in connection with PEM and PC discussed in the previous section. According to Lane, by general propositions Peirce really meant universally quantified propositions. The evidence for this is Peirce's use of the proposition ``Man is mortal'' as an example of a general proposition \citep{lane_peirces_1999}. However, that proposition might be importantly different from the proposition ``All men are mortal.'' In the original proposition, Lane rightly claims that ``Man'' is being used as a ``general propositional subject.'' But this might be because the word ``man'' is a general sign and we would require an exposition of Peirce's semiotic theory to hash out whether this means the same thing as ``All men'', which is more clearly a quantificational subject.

This aside, the reason Peirce gives for why PEM does not apply to general propositions is that ``the general is partially indeterminate.'' In the passage Lane draws this example from, Peirce gives a somewhat different definition of what a general proposition is: ``A sign\dots, that is in any respect objectively indeterminate (i.e., whose object is undetermined by the sign itself) is objectively \textit{general} in so far as it extends to the interpreter the privilege of carrying its determination further. Example: `Man is mortal.' to the question, What man? the reply is that the proposition explicitly leaves it to you to apply its assertion to what man or men you will.'' (CP 5.447) There is a footnote in this passage in which Peirce does mention universal propositions only to note that they are distributively general rather than collectively general. The significance of this is unclear but it does seem to connect generals to universals. Although, the definition he gives in the above passage makes no mention of ``universally quantified propositions.'' 

While there may be difficulties with interpreting generals straightforwardly as universally quantified propositions, Lane takes this and Peirce's statement about PEM to imply that Peirce had a non-standard view of PEM. He introduces two modes in which Peirce may have understood PEM: a material mode and a formal mode.\footnote{This distinction is due to Carnap.} According to the material mode, PEM States: ``for any property and for any individual, either that individual possesses that property or that individual does not possess that property.'' In the formal mode: ``for any pair of contradictory predicates `$P$' and `not-$P$' and for any individual (non-general) subject-term `$S$', either `$S$ is $P$' or `$S$ is not-$P$' is true. (The evidence that Peirce might have thought of PEM this way comes from MS 611 and CP 1.434.) So, on Peirce's understanding, PEM applies only to individual subjects, not generals like properties or predicates.\footnote{So, put simply, where we would normally interpret PEM with respect to general propositions as stating `$\forall x P(x) \lor \neg \forall x P(x)$,' Peirce would have interpreted it as stating `$\forall x P(x) \lor \forall x \neg P(x)$' which is clearly very different.} According to Lane, this is why Peirce claims PEM is sometimes false with regard to general propositions. When the subject of a proposition is general rather than individual, PEM can fail, as in Lane's example ``All Floridians are Miamians or All Floridians are non-Miamians" \citep{lane_peirces_1999}.

Peirce seems to have understood PC in essentially the same way and Lane goes on to spell out the material and formal modes for it as well. What this all amounts to is basically that Peirce's claim that PEM does not apply to generals in conjunction with his idiosyncratic understanding of the principle does not pose a threat to our standard way of thinking about PEM.

Because of his non-standard way of understanding PEM and PC, Lane claims Peirce's denial of PEM applying to generals (or necessities) and of PC applying to vague propositions (or potentials) does not involve any denial of bivalence. This is true since if PEM and PC are restricted to individuals, this does not mean that propositions about these individuals are anything other than true or false. Peirce also claims that PEM and PC not apply necessary and possible propositions for the same reasons they do not apply to generals and vague propositions. Lane takes this lack of conflict with bivalence to demonstrate that Peirce's triadic logic had nothing to do with modality. The argument can be restated as follows: 
\begin{enumerate}
\item Triadic logic must involve a rejection of bivalence. 
\item The denial of PEM applying to necessary propositions and PC applying to possible propositions does not involve a rejection of bivalence.
\item So, Triadic logic could not have been motivated by modality.
\end{enumerate}
While Peirce does not explicitly mention PB anywhere on the pages about triadic logic, it must have been rejected, since any logic that admits a third truth value inherently denies bivalence. Nonetheless, this argument is not valid. The reason for this is that, while Peirce's rather odd understanding of PEM and PC do not require a resection of bivalence it also does not necessitate its acceptance. So, even though Peirce's understanding of modality, does not require rejecting bivalence, this alone is not enough to justify the claim that modality had nothing to do with it.

Aside from the invalidity of this argument, if we were to accept it this would create some explanatory gaps in our account. How would we explain Peirce's use of the phrase ``modes of being'' which he almost certainly understood in modal terms at the time of writing? Furthermore, why would he have put modality in the title of \href{https://iiif.lib.harvard.edu/manifests/view/drs:15255301$639i}{seq. 639} (\ref{fig:639r}) if modality had nothing to do with his project? For these reasons I think we think we should reject Lane's negative point and continue to entertain the possibility that modality was a motivating factor for triadic logic. Lane goes on to argue further that when Peirce was conducting his triadic experiments that he was concerned with propositions of which PEM applies, but is nonetheless false with regard to, returning to the distinction he made earlier. But this would seem to fly in the face of Peirce's claim on \href{https://iiif.lib.harvard.edu/manifests/view/drs:15255301$645i}{seq. 645} (\ref{fig:645}) that PEM is ``not necessarily false'' in regard to triadic logic. It seems more likely that he would say PEM is not false about those propositions, but that it is merely `$L$'.

 The issues with Lane's negative claim by no means undermine the value of his positive claim: that Peirce was motivated by issues surrounding continuity. Continuity was a major concern in Peirce's philosophy throughout the entirety of his life, and especially in this later period. He came to it not only through his own philosophical lens, but also from reading other influential figures, like Cantor,  Dedekind, and Bolzano (more on this in the next chapter). Lane comes to this contribution by exploring an aspect of seq. 645 (\ref{fig:645}) that Fisch and Turquette almost entirely neglect: his inkblot example.
 
 The inkblot example is the only explicit reference to the kinds of propositions the value $L$ is supposed to be assigned to. The example restated reads: 
\begin{quotation}
\noindent``Thus, a blot is made on a sheet. Then every point of the sheet is unblackened or blackened. But there are points on the boundary line; and those points are insusceptible of being blackened or of being unblackened, since these predicates refer to the area about $S$\footnote{$S$ here is the subject of the proposition, the boundary line.} and a line has no area about any point of it.''
\end{quotation}
\noindent At face value, Peirce's example appears to be concerned with cases of predication involving category mistakes, however Lane's analysis reveals that it is more likely about breaches of continuity. Lane claims that the individual subject being Considered here is the boundary line, $B$. The proposition `$B$ is black' then would relieve the valuation $L$ because it is not true that $B$ is black, nor that $B$ is not. Rather, Lane claims quoting Peirce, $B$ ``is at the limit between [black] and non-black]" (Lane, 1999). $B$ in this example, constitutes a breach of continuity. It interrupts at the continuous space on the sheet of paper, and lacks the properties possessed by the regions on either side. Given the importance Peirce placed on continuity elsewhere in his philosophy, this is a much more natural interpretation of the example than category mistakes (although calling $B$ black is, in a sense, still a kind of category mistake on this view).
 
 It might seem odd that Peirce would place so much importance on such a narrow range of propositions that he thought it necessary to revise logic to accommodate them. But Peirce thought continuity was of the utmost importance, and so breaks in continuity would also have been important to him. To illustrate this importance, Lane draws our attention to this passage:
 \begin{quotation}
\noindent``[T]he idea of continuity, or unbrokenness\dots plays a great part in all scientific thought, and the greater the more scientific that thought is; and it is the master key which adepts tell us unlocks the arcana of philosophy'' (CP 1.163).
\end{quotation}
\noindent Peirce's emphasis on continuity in philosophy and in science stems from another doctrine that he endorsed, not altogether unlike the previously mentioned doctrine of tychism. He called this synechism, and says it is the ``doctrine that all that exists is continuous'' (CP 1.172). His reasons for endorsing synechism are far beyond the scope of this thesis, but he likely perceived the continuity of time, space, and thought as evidence supporting his endorsement.

There is further textual evidence to support the link between the inkblot example and Peirce's views on continuity. He used a similar example in the eighth of his Cambridge Conference lectures in 1898 (CCL, all of which are presented in \cite{peirce_reasoning_1992}), which Lane quotes from, where he is quite clearly talking about continuity and boundary properties:
\begin{quotation}
\noindent``I draw a chalk line on the board\dots what I have really drawn there is an oval line. For this white chalk-mark is not a line, it is a plane figure in Euclid’s sense---a surface, and the only line there is the line which forms the limit between the black surface and the white surface\dots the boundary between the black and white is neither black; nor white, nor neither, nor both.'' (CP. 6.203)\label{laneq}\footnote{I present this passage here as Lane does in his paper. We will soon see, briefly in this chapter and extensively in the final chapter, there are a lot of relevant details in the ellipses.}
\end{quotation}
\noindent The inkblot example seems pretty clearly to be a brief expression of the ideas Peirce was elaborating here. Thus, continuity must have factored into the motivations behind triadic logic. While it is apparent continuity was part of the picture; it is not at all clear why this would call for a revision of logic. To help answer this question, Lane tracks the evolution of Peirce’s understanding of continuity and his definition of continua across four stages throughout the course of his life.

Lane places the first stage in Peirce's understanding of continuity from the early years of his career until 1889. During this stage Peirce tells us he understands continuity the same way Kant does. He claims, ``a continuum is precisely that, every part of which has parts.'' (CP 5.335 and 3.256). During this lengthy period, we can also find examples of the properties Lane has dubbed boundary properties. In 1867 Peirce  writes, ``[i]t is not really contradictory... to say that a boundary is both within and without what it bound's” speaking specifically of geometrical boundaries. He gave up on this definition in 1889, marking Lane's stage two, when he took on Cantor's definition because it was ``less unsatisfactory'' (CP 6.164).

Two years later, Peirce reverts back to his original definition but modifies it by adding another property to it. He calls this property Aristotelicity and defines it as follows: ``a continuum contains the end point belonging to every endless series of points which it contains'' (CP 6.123). This definition is somewhat confusing but it appears that Peirce was thinking of the continuum as being made up of intervals. Here too, he claims that continuity and the continuum are defined by the properties of Aristotelicity and Kanticity, where Kanticity is the property of a series that every two points has a point between them. He elucidates this definition with explanations pertaining to the real numbers and infinitesimals. This understanding persisted until about 1898.

Eventually Peirce revised his definition again after noticing a deficiency with the definition of Kanticity, marking Lane's stage four. This change is most clearly stated in a marginal note in Peirce's personal copy of \textit{The Century Dictionary} (CP 6.166-168). Here he tells us that his definition results from a misunderstanding that Kant himself fell into: ``He himself and I after him, understood that to mean infinite divisibility, which plainly is not what constitutes continuity since the series of rational fractional values is infinitely divisible but is not by anybody regarded as continuous''(CP 6.168). Thus, it seems infinite divisibility is not itself enough to make a series continuous. The reason Peirce mentions the series of ``rational fractional values'' here is because between every pair of these there are real and rational values that are not part of the series. In the marginal note Peirce goes on to say, ``[t]he precise definition is still in doubt; but Kant's definition, that a continuum is that of which every part has itself parts of the same kind, seems to be correct.'' (Ibid). This, however, Peirce insists must be understood differently from the mistaken interpretation of continuity as infinite divisibility. In the same passage Peirce tells us that a true continuum wouldn't actually contain any points at all: ``a line, for example, contains no points until the continuity is broken by marking the points. In accordance with this it seems necessary to say that a continuum, where it is continuous and unbroken, contains no definite parts; that its parts are created in the act of defining them and the precise definition of them breaks the continuity'' (Ibid). It is not clear why Peirce thought this as of yet and this will be explored in the following chapter. However, the understanding expressed here seems to have persisted relatively unchanged until the end of his life, as Lane claims.

At this point it might seem like we have two dissonant fragments of explanation as to why Peirce invented his triadic logic. On the one hand, his comments about ``modes of being'' on \href{https://iiif.lib.harvard.edu/manifests/view/drs:15255301$645i}{seq. 645} (\ref{fig:645}) seem to be connected to modality, albeit in an idiosyncratic way, as Fisch and Turquette suggest. On the other, his inkblot example on the same page seems to be connected to his views about continua while having nothing to do with modality, as Lane contends. It might be difficult to see how these fragments could be united, but I argue that, in fact, they can be. At some point it is clear that Peirce came to think of continuity in modal terms.

In 1898, in his Cambridge lectures, he states:
\begin{quotation}
\noindent``a continuum is a collection of so vast a multitude that in the whole universe of possibility there is not room for them to retain their distinct identities; but they become welded into one another. Thus, the continuum is all that is possible, in whatever dimension\footnote{``Dimension'' here and in the next quoted passage may be a technical term. This logical notion is only used by Peirce and a handful of other Boolean logicians\citep{dipert_life_1994}. In the logic of relations, Peirce and his students sometimes thought of propositions as two dimensional. Roughly, these dimensions are the objects propositions range over or are true for and the ``times'' at which it is true (Ibid). Peirce disliked the use of `time' here and preferred to speak of ``possible situations'' (Ibid).} it be continuous'' (NEM 4: 343).
\end{quotation}
\noindent  Notice here Peirce's mention of ``the universe of possibility.'' This is likely to be connected to the special Universes mentioned on seq. 639 (\ref{fig:639r} and to the three Universes he discusses earlier in his notebook, on \href{https://iiif.lib.harvard.edu/manifests/view/drs:15255301$550i}{seq. 550} (\ref{fig:639r}).

There are other examples of Peirce using modal phrases in connection with continuity. The chalkboard example that seems to be another version of the inkblot example also makes use of these. The illustration begins:
\begin{quotation}
 \noindent``Let the clean blackboard be a sort of diagram of the original vague potentiality, or at any rate of some early stage of its determination. This is something more than a figure of speech; for after all continuity is generality\dots This blackboard is a continuum of two dimensions, while that which it stands for is a continuum of some indefinite multitude of dimensions. This blackboard is a continuum of possible points; while that is a continuum of possible dimensions of quality, or is a continuum of possible dimensions of a continuum of possible dimensions of quality, or something of that sort'' (CP 6.203).
\end{quotation}
\noindent In this passage Peirce is describing his hypothetical cosmology based on the doctrine he called ``synechism.'' I will discuss both of these ideas in chapter four. For now, I am simply trying to close the gap between our two avenues of explanation.

There is one final passage I would like to draw attention to in regards to the connection between modality and continuity, one that Lane himself cites. It comes from an unfinished essay entitled ``A Sketch of Logical critic'' written in 1911 for inclusion in a collection intended to honour Lady Welby: 
\begin{quotation}
``Personally, I agree entirely with James against Dedekind's view; and hold that there would be no actually existent points in an existent continuum, and that if a point were placed in a continuum it would constitute a breach of the continuity. Of course, there is a possible, or potential, point-place wherever a point might be placed; but that which only maybe is necessarily thereby indefinite, and as such, and in so far, and in those respects, as it is such, it is not subject to the principle of contradiction, just as the negation of a may-be, which is of course a must be, (I mean that if `$S$ may be $P$’ is untrue, then `$S$ must be non-$P$’ is true), in those respects in which it is such, is not subject to the principle of excluded middle'' (CP 6:182).
\end{quotation}
\noindent This passage seems pretty clearly to establish a connection between Peirce's thoughts on triadic modality and the continuum. He goes on to remark that ``logic here seems to touch metaphysics'' (Ibid). From these examples we can see that these notions were connected in Peirce's thought for at least 13 years.
 
Lane's paper makes important strides in explaining the connection between Peirce's triadic logic and his views on continuity. However, his claim that Peirce's motivation had nothing to do with modality seems to be mistaken.\footnote{Some have even attributed to him ``a \textit{modal logical view of set theory}'' \citep{Putnam1995-PUT}.} It is clear now that modality, triadic logic, and continua were all connected in Peirce's mind. In the next section, I will explore why he thought this.
