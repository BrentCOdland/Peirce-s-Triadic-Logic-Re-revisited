\chapter{Continuity, Modality, and Cosmology}
%At the end of the previous chapter, I made reference to a number of passages from Peirce, while giving little explanation as to what these actually meant. The aim of that discussion was to establish that Peirce could have been motivated by both triadic modality and continuity, contrary to Lane's thesis. In this chapter many of those passages will be rendered clear. 

So far I have shown that viewing Peirce's motivations for triadic logic as due exclusively to either modality or continuity is mistaken. These two notions were intimately connected for him. In this chapter I will flesh out the details of this connection and show how modality and continuity meet under Peirce's hypothetical cosmology. I will do this by answering a handful of questions more specific than the broad scoped question as to what was motivating Peirce. These are: 1. What was Peirce trying to formalize or bring into the scope of logic? 2. Why did he need his third truth-value, L, to do this? And 3. What propositions take the value L? 

To answer these questions, I will draw mainly from an unpublished manuscript entitled ``A Fourth Curiosity'' (AFC), which was written in the same year Peirce conducted his three valued experiments, as well as CCL. The former was originally intended for publication in \textit{The Monist} as part of a series Peirce was writing called ``Amazing Mazes” \citep{peirce_amazing_1908}. A handful of these were published but AFC and another paper called ``A Third Curiosity'' were not.

AFC is an odd document in part because of its wide range of subject matter and also because it is not altogether clear what the point of it is. It begins with discussion of the logic of relations, then moves on to philosophical and mathematical notions of time, then modes of being, before finally ending with discussions of cardinality and infinity, making reference to Cantor, Dedekind, and Bolzano. Many of these are topics that have come up in our previous discussion. My hope is that, given that Peirce may have been writing this document at more or less the same time as he wrote of triadic logic in his notebook, his explication of these similar topics will provide insight into what he meant in his notes.

My discussion in this chapter will rely heavily on Peirce's ``modes of being'' and his ``special universes.'' To help keep matters straight, it will be helpful now to briefly re-explain these notions against the background of another triad in his philosophy, of which these other tripartite divisions are an offshoot: his three Universal Categories, some version of which was endorsed as far back as \citeyear{peirce_five_1865}, when he argues that these categories are necessary to give a unified concept of experience.

Peirce had three categories that he used in his analysis of all philosophical ideas. He sometimes calls them categories of being, sometimes of ideas, of thoughts, and of nature. Depending on the subject matter, these categories will have different names, but in the most abstract sense they are always the same. In the abstract, the categories are \textit{firsts} (things that are what they are without reference to anything else), \textit{seconds} (things that exist only in reference or connection something else), and \textit{thirds} (which exist in bringing together a second and a third). An example of the sort of thing that would be a first for Peirce is a quality, like a color. There is some sense in which colors exist without reference to anything else. We all have a concept of `redness' that we can think of independently of red objects. So the color red is a first. Seconds are like particular objects, like a red blanket. A red blanket is what it is in reference to two things: being red and being a blanket. Thirds do not lend themselves to simple examples as easily, and are more recognizable within the various contexts Peirce applies his categories, like his modes of being or special universes. He sometimes calls them laws, sometimes reasons, and other times powers.

Having some idea of Peirce's categories, we can now provide a better explanation of his three modes of being. Each of the objects that falls into his three categories has an associated mode of being: ``My view is that there are three modes of being. I hold that we can directly observe them in elements of whatever is at any time before the mind in any way. They are the being of positive qualitative possibility, the being of actual fact, and the being of law that will govern facts in the future'' (CP 1.23). So, firsts, being merely qualities, have the mode of being of a possibility. This is because they do not exist on their own except for in the possibility that they are instantiated by some object. Seconds, have the mode of being of actuality. These are ordinary objects that occur in the world around us. All the objects that we normally see and interact with are seconds. Thirds have the mode of being of necessity. Again, this notion is much more difficult to understand as precisely as the first two. The things that are thirds for Peirce can loosely be understood as laws of nature. They are basically general facts about seconds. He says ``This mode of being which \textit{consists}, mind my word if you please, the mode of being which \textit{consists} in the fact that future facts of Secondness will take on a determinate general character, I call a Thirdness'' (CP 1.26). Part of the difficulty with understanding and stating precisely what thirds are is thinking of them as objects. We do not normally think of laws as objects. Nonetheless, suppose that every time a diamond is dragged across a pane of glass, from now into the indefinite future, a scratch is produced. Then the third in this case is the fact about the universe that makes it so that all diamonds scratch all panes of glass.

Now, as we have already seen, in addition to three categories and three modes of being, Peirce also proposed a notion of three universes \href{https://iiif.lib.harvard.edu/manifests/view/drs:15255301$550i}{(seq. 550)}. One consisting of ideas, one of occurrences, and another of powers. Each of these three universes has one of Peirce's modes of being \href{https://iiif.lib.harvard.edu/manifests/view/drs:15255301$552i}{(seq. 552)}. The universe of ideas has the mode of being of possibilities. The universe of occurrences has that of actuality. Finally, the universe of powers has the mode of necessity. The work that these three universes do is primarily within Peirce's hypothetical cosmology. This will be detailed further in the final section of this chapter but simply put, Peirce believed (or at least entertained as a hypothesis) that the universe begins in a vacuous state of mere possibility, a universe of firsts. From there it evolves to the existing universe, where some of the possibilities in the initial state have become determined. Peirce tells us that this evolutionary process is one where a world of Platonic forms is incrementally becoming determined: \begin{quotation}\noindent``From this point of view we must suppose that the existing universe, with all its arbitrary secondness, is an offshoot from, or an arbitrary determination of, a world of ideas, a Platonic world; not that our superior logic has enabled us to reach up to a world of forms to which the real universe, with its feebler logic, was inadequate... The evolutionary process is, therefore, not a mere evolution of the existing universe, but rather a process by which the very Platonic forms themselves have become or are becoming developed'' (CP 6.192-194).
\end{quotation} The actual universe occupies a space between the universe of possibility and the universe of laws or powers, and is partially determined by each of these two poles. So, the actual universe is partially determined by arbitrary possibility on the one hand, and approaches and is partially determined by the universe of necessity on the other. As this evolutionary process unfolds, the universe becomes incrementally more determinate. Within this process, the three universes appear to be on a continuum of their own.

It might be unclear at the moment what any of what I have just said has to do with Peirce's triadic logic. However, it is at the intersection of these ideas where we find that our two clues, the modes of being and the continuity example, connect.

This chapter will be divided into three sections. The first will be aimed towards explaining what Peirce was trying to represent. In the second and third sections I will answer the related questions of why Peirce needed his third value L, and what propositions would recieve the value L.

%1. What was Peirce trying to formalize or bring into the scope of logic? 2. Why did he need his third truth-value, L, to do this? And 3. What propositions take the value L? 


\section{What was Peirce trying to represent?}

One thing that can be gleaned from our examination of the notebook pages is that Peirce seemed to have an idea of what he wanted to extend logic to represent, before he knew exactly how to do so. This is evident from the change of approaches he makes in that cluster of pages from his logic notebook (seq. 637--645). He begins by adding a new operator to his existential graphs, then switches back to a traditional symbolic notation, and alternates between two and three-valued semantics before he seems to have decided on seq. 645 (\ref{fig:645}) that adding a truth value was the correct approach. So, what was Peirce trying to represent? Furthermore, why was classical logic incapable of representing it?

Part of the reason classical logic is not up to the task has to do with the universe it represents. In AFC, Peirce tells us that his previous work on logic has focused on ``existential relations'' which have only to do with one logical universe: \begin{quotation}\noindent``An \textit{existential} relation is distinguished from others by two marks. In the first place, its different subsets all belong to one universe\dots In the second place, an existential relation or relationship differs from some other relations and relationships in a respect which may be described in two ways, according as we employ collective or distributive forms of expression and thought'' (CP 6.318).\end{quotation} This second point of difference is more easily understood when it is expressed according to this collective form. \begin{quotation}\noindent``Speaking collectively, the one logical universe, to which all the correlates of an existential relationship belong, is ultimately composed of \textit{units}, or subjects, none of which is in any sense separable into parts that are members of the same universe. For example, no relation between lapses of time---say, between the age of Agamemnon and that of Homer---can be an existential relation, if we conceive every lapse of time to be made up of lapses of time, so that there are no indivisible units of time'' (Ibid).\end{quotation} Perhaps it is helpful to frame Peirce's discussion here within the special universes of his hypothetical cosmology. Recall that Peirce conceived of three special universes, of ideas, occurrences, and powers or reasons. He also identified these universes with his modes of being, claiming the mode of being of the first universe is possibility, the second actuality, and the third necessity \href{https://iiif.lib.harvard.edu/manifests/view/drs:15255301$552i}{(seq. 552)}. The relations that Peirce is discussing here are located within the second universe, which is actual. According to him, this universe is made up of units, and this means it is not continuous, so these existential relations must be non-continuous as well.

In the following paragraph, Peirce goes on to state that his previous work on logic has been limited to the study of existential relations: \begin{quotation}\noindent``My reasons for mostly limiting the scope of my logical studies of relations were, firstly, that these are very tangible and logically tractable; secondly that the great body of other sorts of relations differ from these merely in being indeterminate in some respects in which existential relations or some species of these are determinate, so that the logical theory of these virtually puts the student into possession of the logical theory of all but a very few recondite relations\dots ''(CP 6.319).\end{quotation} Since the bulk of Peirce's previous logical work is concerned with what we now call classical logic, we can infer that this is the subject he is drawing limits around. Classical logic deals with what he calls existential relations. So, on Peirce's view, classical logic represents a limited class of relations. 

From this, and Peirce's remarks about special universes and modes of being in his notebook, I think we can answer the questions posed at the beginning of this section. Why was classical logic inadequate for Peirce? Based on these passages, it seems he thought it was only capable of representing relations in one of his ``special universes.'' More specifically, it is only capable of representing his second Universe, ``of occurrences (existent things and actual events)'' (MS 339, \href{https://iiif.lib.harvard.edu/manifests/view/drs:15255301$550i}{seq. 550}). So, it seems that in conducting his three-valued experiments, Peirce was trying to extend logic so that it could adequately represent the other two universes, which are modal in nature. There are a couple of reasons given here for why classical logic was not up for this task. One is that it treats relations that are indeterminate as determinate (CP 6.319). The second has to do with classical logics subject being ``ultimately composed of \textit{units}” (CP 6.318). Because of this, classical logic is apparently incapable of representing relations that are continuous, as is evident from Peirce's time example. Because classical logic was apparently incapable of expressing continuous relations, this also meant it was incapable of representing universes other than the second.\footnote{This does not necessarily mean that Peirce thought classical logic incapable of expressing truths about continuous \textit{domains}. This restriction appears to only apply to relations.} This is why Peirce says, on seq. 645 (\ref{fig:645}), that dyadic (or classical two-valued) logic is not false or incorrect, only that it is limited.

A similar point could be made regarding Peirce's ``modes of being,'' which correspond to his universes. Classical logic, being restricted ``to the existential class'' of relations, must also be restricted to the existential mode of being, which Peirce characterizes as follows: ``In the metaphysical sense, existence, is that mode of being which assists in the genuine dyadic relation of a strict individual with all the other such individuals of the same universe'' (CP 6.336). Here Peirce explicitly connects his ``modes of being'' to his logical universes. Given his characterization of existential relations at the beginning of AFC, as well as his comment that most of his work on logic has been restricted to those relations, we can infer that on his view classical logic would have been restricted to the existential mode of being as well. 

Peirce's discussion of the other modes of being will help us determine what is missing, given this restriction. He writes: \begin{quotation} \noindent``so, then, there are these three modes of being: first, the being of a feeling, in itself, unattached to any subject, which is merely as atmospheric possibility\dots; secondly, there is the being that consists in arbitrary brute action upon other things\dots and thirdly, there is living intelligence from which our reality and power are derived: which is rational necessity and necessitation'' (CP 6.342).\end{quotation} He elaborates on these further: \begin{quotation}\noindent``A feeling is what it is, positively, regardless of anything else. Its being is in it alone, and it is a mere potentiality. A brute force, as, for example, an existent particle, on the other hand, is nothing for itself;\dots its being is actual, consists in action, is dyadic. That is what I call existence. A reason has its being in bringing other things into connexion with each other; its essence is to compose: it is triadic, and it alone has a real power'' (CP 6.343).\end{quotation} So, Peirce identifies the existential mode of being with the second of these. Then, what is missing if classical logic is restricted to this mode, are the other two. Consequently, the reason classical logic was not up to the task is that it is incapable of representing potentiality or necessity.\footnote{In his subsequent discussion of modes of being in ``A Fourth Curiosity,'' Peirce speaks to the ranking that is implied by his remarks in the first paragraph of seq. 645 (\ref{fig:645}). He explains what he means by a ``lower mode of being,'' but in such a cryptic way I cannot make much sense of it. After some brief remarks about \textit{blind existential being} and mention of a book on God and religion he intends to write, he claims: ``This much, however, seems clear about such existence: namely that there ought to be two grades of it; a lower kind, approximating to the inner being of a simple quality, yet existential, instead of being merely potential, consisting in the action of a thing upon all the other things of the same universe, and measuring by its intensity its remoteness from each of them. A whole universe of such existents can only have the lower, or internal grade of existence'' (CP 6.346).} 

I believe here we have an answer to our first question. In his notebook pages Peirce was attempting to extend logic to be capable of representing features of his other two universes, or his other two modes of being. These seem to be connected to alethic modalities in some way. However, to say that he was trying to create a simple modal logic like we are familiar with today would be an oversimplification. Peirce had already created a modal logic by this point, by adding a modal operator to his beta graphs, without resorting to abandoning bivalence (CP 4.510--529 and 573--584).\footnote{J. Zeman has proven in an 1964 \href{http://users.clas.ufl.edu/jzeman/graphicallogic/gamma.htm}{unpublished doctoral thesis} that Peirce's modal graphs are equivalent to S4 and S5 \cite{zeman1964graphical}.} Nevertheless, some clues might be found by examining his comments about the limitations of existential relations. Specifically, his comment that the logical universe these relations pertain to ``is ultimately composed of units'' and his time example which would seem to illustrate existential relations cannot be continuous. Given that classical logic seems to be restricted to this class of relations this might explain why Peirce says that dyadic logic ``is not absolutely false, it is only $L$'' (seq. 645, \ref{fig:645}). It is limited because it cannot capture continuity.

\section{Why did Peirce need L?}

The next question is more difficult to answer than the first. We have more or less established that Peirce deviates from dyadic logic because he thought it incapable of expressing propositions about anything other than actuality. This is apparently because dyadic logic is incapable of representing continuous relations, since the universe of dyadic logic ``is ultimately composed of units, or subsets...“ (CP 6.318). But why did he feel he needed to switch to three values to go beyond this? I think the answer has to do with his view ``that there would be no actually existent points in an existent continuum''(CP 6.182). He claims instead, that ``there is [only] a possible, or potential, point-place wherever a point might be placed; but that which only maybe is necessarily thereby indefinite'' (Ibid.). This may be the reason for including the value $L$. If points on the continuum are only possible, or indefinite, then there must be propositions attributing properties to these objects, that would most appropriately be assigned the value $L$. This may also help us understand Peirce's remark about a ``lower mode of being'' on seq. 645 (\ref{fig:645}). Points on a continuum, being merely possible, would have a lower mode of being than anything actual, like an actual continuum itself.

But why did Peirce take this view of continua? It is not yet clear what made him think that there would be no points on a continuum. Returning to a passage Lane brings up in his article, I think we can approach an answer: ``a continuum is a collection of so vast a multitude that in the whole universe of possibility there is not room for them to retain their distinct identities; but they become welded into one another'' (NEM 4: 343). Peirce's remarks here are somewhat opaque, but I would conjecture that what he really means by this is that real continua are uncountable. That is why ``there is [no] room for its individual members to retain their distinct identities.'' Thus, part of the reason Peirce came to view continua so, might be a reaction to Cantor's uncountability results. 

%you found a nice piece discussing versos and rectos regarding sheets of assertion and modality in EG. The rectos are for the actual and the versos on the other side of the same sheet are possible. It is also interesting that the next page with writing on it following 645 is about tinctures, which is where Peirce talks about rectos and versos.
Peirce gives Cantor some rather odd praise in ``A Fourth Curiosity.'' He says, ``Dr. Georg Cantor, of Halle, undertook that research which I have mentioned as of the greatest urgency for logic, for metaphysics, and for cosmogony, that of ascertaining whether or not the singulars of every collection, however great, can be the subjects of a linear relation, and if not what is the greatest multitude of singulars that can be so arranged'' (CP 4.675). While Cantor's interpretation of his own results did apparently lead him to certain metaphysical and religious commitments, he might have been surprised to learn of his contributions to the field of cosmogony\footnote{The study of the origins of the universe. Elsewhere Peirce usually uses the word ``cosmology" for what seems to amount to the same thing.}. The work that Peirce is referring to here has to do with techniques for counting the members, or cardinality (Peirce uses the word `multitude' where we would normally speak of cardinality), of finite or infinite sets. The ``linear relation'' he mentions can be understood as establishing whether or not a set can be put into a well ordered list, with a first member and every other member of the set appearing in a place on that list. Some infinite sets, like the natural numbers, can be put in this kind of ordering. The real numbers, which would probably have been the best example of a continuum for Peirce, however, can not be ordered in such a way because for every two numbers in the set, $x$ and $y$, there will always be some number $z$ between them, no matter which members $x$ and $y$ designate. So, it is impossible to put the real numbers on such a list without missing members.\footnote{Peirce gives a charming explanation of another similar result proven by Cantor: that the powerset (the set of all subsets of a set) of any set has greater cardinality than the set itself. ``Let there be a denumerable collection, say the cardinal numbers; and let there be two houses. Let there be a collection of children, each of whom wishes to have those numbers placed in some way into those houses, no two children wishing for the same distribution, but every distribution being wished for by some child. Then, as Dr. George Cantor has proved, the collection of children is greater in multitude than the collection of numbers. Let a collection equal in multitude to that collection of children be called an abnumeral collection of the first dignity. The real numbers (surd and rational) constitute such a collection.

I now ask, suppose that for every way of placing the subjects of one collection in two houses, there is a way of placing the subjects of another collection in two houses, does it follow that for every subject of the former collection there is a subject of the latter? In order to answer this, I first ask whether the multitude of possible ways of placing the subjects of a collection in two houses can equal the multitude of those subjects. If so, let there be such a multitude of children. Then, each having but one wish, they can among them wish for every possible distribution of themselves among two houses. Then, however they may actually be distributed, some child will be perfectly contented. But ask each child which house he wishes himself to be in, and put every child in the house where he does not want to be. Then, no child would be content. Consequently, it is absurd to suppose that any collection can equal in multitude the possible ways of distributing its subjects in two houses'' (CP 3.547-548, introductory chapter for \textit{Minute Logic}, 1902).} 

This result is likely the reason why Peirce says that there is not enough room for points on a continuum to ``retain their distinct identities'' and why he says these points become ``welded together'' (NEM 4:343). That Peirce derived this property from the uncountability of the continuum has already been argued by Moore \citeyearpar{moore_genesis_2007}. But what does the continuum having these properties have to do with modality? 

From an early point in his career, Peirce began equating continuity with generality which, as we have seen above, has a modal interpretation for him. What's more is he seems to have thought of the universe, or ``field,'' of possibility as continuous and hence uncountable as well (NEM 4:358). In a rather confused passage in CCL, he links the uncountability and lack of discrete individuals in a continuum with his view that it would only contain possible points:
\begin{quotation}
\noindent ``...the series of abnumeral multitudes... will have a limit if there is properly speaking any meaning in saying that something that is \textit{not} a multitude of distinct individuals is \textit{more} than every multitude of distinct individuals. But, you will ask, can there be any sense in that? I answer, yes, there can in this way. That which is possible is in so far \textit{general}, and as general, it ceases to be individual. Hence, remembering that the word ``potential'' means \textit{indeterminate yet capable of determination in any special case}, there may be a \textit{potential} aggregate of all the possibilities that are consistent with certain general conditions; and this may be such that given any collection of distinct individuals whatsoever, out of that potential aggregate there may be actualized a more multitudinous collection than the given collection. Thus the potential aggregate is with strictest exactitude greater in multitude than any possible multitude of individuals. But being a potential aggregate only, it does not contain any individuals at all. It only contains general conditions which \textit{permit} the determination of individuals'' (CP 6.185).
\end{quotation}
\noindent So it seems that Peirce reasoned that the points on continua were not discretely identifiable because continua are uncountable, and this seems to be the reason it would contain only possible individuals.

While none of this would obviously necessitate that Peirce introduce a third truth value to logic, it may provide a reason why he thought he should. The indeterminacy involved in his conception of the continuum is probably the reason that he thought it necessary to introduce $L$. Because the points on the continuum are indeterminate and merely potential on his view, it seems likely that there could be properties that are neither true nor false of these objects. However, it still is not exactly clear what those properties might be.

\section{What propositions are L?}

The third question is by far the most difficult to answer. Since the continuum of reals is probably the most serviceable example for Peirce's analysis of continuity, we might expect that there would be propositions about the real numbers that would take the value L, at least according to his notion of a continuum. Sadly, it is difficult to imagine what these mathematical examples would be or even if there are any. The only obvious examples of $L$-propositions seem to be the boundary propositions that Lane has identified from the inkblot example.
\begin{quotation}
\noindent``Thus, a blot is made on a sheet. Then every point of the sheet is unblackened or blackened. But there are points on the boundary line; and those points are insusceptible of being blackened or of being unblackened, since these predicates refer to the area about S and a line has no area about any point of it.''
\end{quotation}
\noindent Even though Lane has helpfully connected this passage to Peirce's views on continuity, it still is not obvious why such a specific range of propositions would warrant this revision.\footnote{At least when he certainly was aware of more general issues, like modality and vagueness.} Luckily, Peirce used this kind of example to illustrate points about continuity more than just the couple of times we have seen so far and some of these are much more illustrative of the philosophical significance of boundary propositions. It appears that Peirce may have had a slightly different continuum in mind when he gave the inkblot example, and he was thinking more along the lines of time rather than the reals.

In his paper, ``The Law of Mind'', published in \textit{The Monist} \citeyear{peirce1892law}, Peirce makes use of this kind of example again:
\begin{quotation}
\noindent``Suppose a surface to be part red and part blue; so that every point on it is either red or blue, and, of course, no part can be both red and blue. What, then, is the color of the boundary line between the red and the blue? The answer is that red or blue, to exist at all, must be spread over a surface; and the color of the surface is the color of the surface in the immediate neighborhood of the point. I purposely use a vague form of expression. Now, as the parts of the surface in the immediate neighborhood of any ordinary point upon a curved boundary are half of them red and half blue, it follows that the boundary is half red and half blue. In like manner, we find it necessary to hold that consciousness essentially occupies time; and what is present to the mind at any ordinary instant is what is present during a moment in which that instant occurs. Thus, the present is half past and half to come.''
\end{quotation}
\noindent This passage, not cited by Lane, is clearly relevant to how this example is used in the notes on triadic logic. Here Peirce gives a bit more indication as to why the boundary point is neither blue nor red (or blackened or unblackened in the familiar version on seq. 645, \ref{fig:645}). It is apparently because that property belongs to a point based on what is immediately next to it and on a boundary line this would take on both of the properties on either side. When the properties on either side are contradictory, as with black and not black, then it seems reasonable to say that propositions attributing either property to points on the boundary line would have to be $L$ because they would have both in some sense. 

Peirce's connecting this example to time is helpful to see why he thought these kinds of propositions were significant. When the continuum is time, it appears he thought the present acted as the boundary line between the past and the future. So presumably there are propositions about the present that would rightly be assigned the value $L$. If Peirce was thinking about continuity and time, that would explain the inclusion of the word `temporal' in the title of the seq. 639 (\ref{fig:639r}. But what kind of temporal propositions would take the value $L$? One common feature in all the various versions of this example is that on either side of the boundary there is a property present that the other side lacks. These properties are usually such that a particular object cannot have both. For Peirce, when the past and future are what is separated, the property possessed on one side but lacking on the other is that of determinacy. On his view, the past is wholly determinate while the future is indeterminate. Following through with the other examples, a proposition asserting that the present has either of these properties would be assigned the value $L$.

To spell this out further and show why Peirce would have thought this significant, it will be helpful to return to the Cambridge lecture in which he gave the chalkboard version of the boundary example. In the context of the lecture, Peirce was using the example to illustrate his hypothetical cosmology. 
\begin{quotation}
\noindent``Now continuity is shown by the logic of relations to be nothing but a higher type of that which we know as generality. It is relational generality.\footnote{CCL seems to be the first place where Peirce associates continuity with relational generality. By generality of the implied lower type, he means something like \textit{universals} \citep{moore_genesis_2007}. Continuity then is a higher type of generality than universality. It is not clear why he came to this view or exactly what he means by statements such as these. It might be generality over properties rather than individuals.}

How then can a continuum have been derived? Has it for example been put together? Have the separated points become welded, or what?

Looking upon the course of logic as a whole we see that it proceeds from the question to the answer --- from the vague to the definite. And so likewise all the evolution we know of proceeds from the vague to the definite. The indeterminate future becomes the irrevocable past... However it may be in special cases, then, we must suppose that as a rule the continuum has been derived from a more general continuum, a continuum of higher generality'' (CP 6.190--191).
\end{quotation}
\noindent Here it seems Peirce is trying to draw inferences about continua in general based on special cases. Time is the second of these. Peirce's remarks about the past and future above seem to confirm what I have just said in trying to apply his boundary example to time. This process of going from ``the vague to the definite'' appears to apply to more continua than time as well. All of this is part of Peirce's broad cosmological or metaphysical theory.
\begin{quotation}
\noindent``From this point of view we must suppose that the existing universe, with all its arbitrary secondness, is an offshoot from, or an arbitrary determination of, a world of ideas, a Platonic world;\dots

If this be correct, we cannot suppose the process of derivation, a process which extends from before time and from before logic, we cannot suppose that it began elsewhere than in the utter vagueness of completely undetermined and dimensionless potentiality.

The evolutionary process is, therefore, not a mere evolution of the existing universe, but rather a process by which the very Platonic forms themselves have become or are becoming developed'' (CP 6.192--194).
\end{quotation}
\noindent This passage exposes Peirce's realism. He appears to believe that this aspect of his continua that is more easily understood in temporal form also applies in metaphysics. Just as the future is determined when it becomes the past, it appears Peirce thought something similar is happening with forms as the universe evolves. The forms receive an arbitrary determination in ``the existing universe.'' It seems as though Peirce thought just about everything is involved with this kind of a process in one way or another. In the lecture, he uses these notions to describe his hypothetical cosmology:
\begin{quotation}
\noindent``We shall naturally suppose, of course, that existence is a stage of evolution. This existence is presumably but a special existence. We need not suppose that every form needs for its evolution to emerge into this world, but only that it needs to enter into some theatre of reactions, of which this is one.

 The evolution of forms begins or, at any rate, has for an early stage of it, a vague potentiality; and that either is or is followed by a continuum of forms having a multitude of dimensions too great for the individual dimensions to be distinct. It must be by a contraction of the vagueness of that potentiality of everything in general, but of nothing in particular, that the world of forms comes about'' (CP 6.195-196).
\end{quotation}
What Peirce is attempting to describe is how the universe evolves, which involves some movement from the indeterminate to the determinate. He also thought that this applied beyond our universe, to ``the whole Platonic world'' as well (CP 6.200). One topic discussed in the previous section comes up again in this passage. He says the evolution he is talking about begins with a vague potentiality that consists in ``a continuum of forms having a multitude of dimensions too great for'' its individuals to be distinct. As we have seen before, this probably means that the forms he spoke of are uncountably infinite. Peirce seems to attribute this property to potentiality whenever he discusses it and this is probably because of the vast number of ways things could unfold according to his cosmological theory of evolution. This passage also can help us to understand why he was so interested in boundaries. The boundary lines in Peirce's examples must be analogues of the ``theatre of reactions'' he mentions. This is where ``the vagueness of that potentiality of everything in general, but of nothing in particular'' contracts as he puts it. The boundary is where potentials begin to become determined. In the case of time, the present is the boundary at which the indeterminate future begins to become the determinate past.

Against this background, the blackboard example that Lane cites\footnote{Lane presents this passage in a very stripped down fashion, see section \ref{laneq}.} will make a lot more sense and help us understand how all of this worked in Peirce's mind.
\begin{quotation}
\noindent``Let the clean blackboard be a sort of diagram of the original vague potentiality, or at any rate of some early stage of its determination. This is something more than a figure of speech; for after all continuity is generality. This blackboard is a continuum of two dimensions, while that which it stands for is a continuum of some indefinite multitude of dimensions. This blackboard is a continuum of possible points; while that is a continuum of possible dimensions of quality, or is a continuum of possible dimensions of a continuum of possible dimensions of quality, or something of that sort. There are no points on this blackboard. There are no dimensions in that continuum. I draw a chalk line on the board. This discontinuity is one of those brute acts by which alone the original vagueness could have made a step towards definiteness. There is a certain element of continuity in this line. Where did this continuity come from? It is nothing but the original continuity of the blackboard which makes everything upon it continuous. What I have really drawn there is an oval line. For this white chalk-mark is not a line, it is a plane figure in Euclid's sense — a surface, and the only line there, is the line which forms the limit between the black surface and the white surface. Thus the discontinuity can only be produced upon that blackboard by the reaction between two continuous surfaces into which it is separated, the white surface and the black surface. The whiteness is a Firstness — a springing up of something new. But the boundary between the black and white is neither black, nor white, nor neither, nor both. It is the pairedness of the two. It is for the white the active Secondness of the black; for the black the active Secondness of the white'' (CP 6.203, CCL).
\end{quotation}
It is now apparent that Peirce was using this example to illustrate his cosmological metaphysics. The blank board is the ``vague potentiality'' he mentions earlier in the lecture. The chalk line is a continuity breach that renders this original potentiality more definite. The boundary lines are the ``theatre of reactions'' where the separation between the two becomes determined. It is a ``second'' to the two because it exists only in its relation to them.

This understanding of Peirce's boundary examples makes it easier to come up with propositions of the sort he likely imagined would take the value $L$. The value $L$ still seems to be reserved for boundary propositions however, this might not be as narrow a range of propositions as previously thought. Returning to temporal examples, we might think that whenever some event happens there is always some exact moment at which it occurs. This moment is the boundary between the ``indeterminate future'' (which Peirce would also call a vague potentiality) and the ``irrevocable past.'' It is the moment that determines the actual event from the mere possibility of it. It is a second to each side because it is defined by them. Following Peirce's discussion above, the boundary would have to be partially indeterminate and determinate. Now, because both sides of the boundary are continuous (and possibly because it is partially indeterminate), the boundary itself must be continuous also. Consider the proposition `$x$ was born at $t$.' Because time is continuous, we can always carry this determination further, say by asking whether $x$ was born at $t.1$ or $t.001$, and so forth. Thus, this is a kind of boundary proposition in these regards and would likely be the kind of proposition. 

According to Peirce's generalization, these types of processes apply to continua in general, so it should be possible to construct these kinds of examples for other types of continua as well. However, the difficulty of coming up with mathematical examples of boundary propositions might pose problems for the generality of Peirce's theory. Still, perhaps it is possible to come up with a similar example regarding transcendental numbers. Suppose I say that `the $n$th digit of $\pi$ is $x$', where n is a value that has never been computed so far. If we regard computation as the thing that determines $x$, then this proposition might properly be assigned the value $L$. But this goes far beyond what we can reasonably infer based on the textual evidence.

It appears as though we have answered the questions posed at the beginning of the section. In experimenting with three-valued logic, Peirce was trying to take logic to a higher level of generality. He wanted to elevate it to the same level that he thought he had reached with his metaphysical theories, presumably to learn more about what those theories implied. He wanted to extend logic to be capable of reasoning about the ``Universe of Possibility'' and the ``Universe of Necessity''. Because he thought these three universes were related in a continuous process, he needed to account for the indeterminacy involved to extend logic this way. And this, in turn is because their is inherent fuzziness at the boundary between the determinate and indeterminate. Peirce was motivated by a desire to extend logic to facilitate reasoning about the universe as represented by his hypothetical cosmology.
