\chapter{Introduction \\ \small{Charles Sanders Peirce: Logician, Philosopher, and Beyond}}
Charles Sanders Peirce is probably best known as founder of the increasingly popular school of thought in philosophy  called pragmatism. He was far more obscure in his day and likely would have been completely so if it were not for his close friend, William James, who adapted and popularized some of Peirce's ideas (most notably his Pragmatic Maxim). In fact, Peirce's claim to fatherhood might be contested, as it was James who first brought pragmatism to general audiences.\footnote{Peirce ended up disliking James' version of the doctrine so much that in 1905, he began to call his own version ``pragmaticism," to distance himself from James and other pragmatists (CP 5.414).} While Peirce and James were dear friends, they differed sharply on some issues, especially on the nature of truth as prescribed by their pragmatism. Another area where the two differed was on metaphysics, where James was a nominalist and Peirce an adamant scholastic realist. Peirce was initially skeptical of metaphysics, but as time went on he devoted increasingly more thought to it. By the end of his life he had constructed a grand theory of speculative cosmology, incorporating a rich triadic ontology.

As a philosopher, Peirce was insightful, ambitious, and exceptionally creative. However, his complexity and penchant for inventing new technical terms sometimes led to an underwhelming reception of his work. The fact that the number of published works are dwarfed by the wealth of manuscripts, speculating on all manner of things from religion to logic machines, makes matters difficult for the Peirce scholar. The difficulty of Peirce scholarship is nicely summed up by this anecdote from a book review on a volume produced by Edward Moore and Richard Robin:\begin{quote} \noindent``There is a story that when Peirce retired to Milford, he built an attic study in his house accessible only by ladder. When the creditors that plagued his later years came calling, Peirce retreated to uninterrupted philosophizing by climbing to his study and drawing up the ladder behind him.

This story suggests a moral: the student of Peirce is often in a position like that of the bill collector. Access to Peirce can be difficult'' \citep[73]{thayer_book_1967}.\end{quote}

As this story betrays, Peirce led a troubled life. It is ironic that now we know him primarily as a philosopher, as in his own lifetime he made his living primarily as a working scientist.\footnote{A further irony can be found in the fact that a working scientist would embrace such a flamboyantly speculative metaphysics.} The only academic position he held was his appointment as a lecturer at Johns Hopkins. He held this position from 1879 until 1884, when he was fired for reasons tied to his divorce and subsequent marriage of his second wife Juliette \citep{hoopes_review_1999}. His isolation from the academic community partially explains the difficulty of his philosophy. There were few with whom he could test out ideas with. His student and long time correspondent and co-author, Christine Ladd-Franklin once wrote: \begin{quotation}\noindent``If Charles S. Peirce had happened to have a longer period of activity at the Johns Hopkins University---if the years had not been cut off during which he was kept upon the solid ground of intelligible reason by discussions with a constantly growing group of level-minded students,---there is no doubt that his work would have been of more certain value than it can be affirmed to be now'' \citep{ladd-franklin_charles_1916}.\end{quotation} Despite his tenuous academic career, Peirce remained keenly interested in philosophy, devoting all of his time to it when he was not working on the intellectual odd jobs he used to sustain himself.

The subject which he was hired to lecture on and which he taught Ladd-Frankin, O.H. Mitchell, John Dewey, and others, was logic. Peirce has been somewhat pushed out of the history of logic by the towering presence of Frege and Russell. His importance in the history of logic was, for a long time, not well understood until the  pioneering work on the subject by \citet{Dipert1995-DIPPUP} and \citet{hintikka_place_1997}  was published. As a logician, Peirce worked within the algebraic tradition, which finds its roots in Boole's algebra of logic. He and Mitchell, discovered quantification independently of Frege. It is unclear when Peirce made this discovery, but it was no later than 1883 \citep{mitchell_new_1883}.\footnote{Interestingly, at the time quantification was apparently more obvious an advancement than we typically understand now. Frege never especially touted this achievement, nor does he use the phrase `quantification' to describe any aspect of his logic \citep{Dipert1995-DIPPUP}. Peirce seems to be the first person to specifically discuss quantification \citep{putnam_peirce_1982}.} He showed that propositional logic is expressively complete under a single operator, which we now refer to as ``Peirce's Arrow.'' Furthermore, the notation he and his pupils used was appreciated and expounded by Schröder and is only a typographical variant of the notation logicians currently use \citep{putnam_peirce_1982}. Löwenheim proved his famous theorem in this notation; and Zermelo wrote his axiomatic set theory in it (Ibid).

Peirce also appears to be the first person to distinguish between first and second order logics (Ibid); and even attempted to work with non-classical logics to deal with issues like modality. This brings us to the topic of my thesis.

In 1909, from around January 7th to February 23, Peirce began experimenting with three-valued logic, anticipating the pioneering work on the subject by  Łukasiewicz (\citeyear{Lukasiewicz1920}, translated in \citeyear{Lukasiewicz1970a}) and Post (\citeyear{post_introduction_1921}) by about 10 years. Łukasiewicz and Post arrived at their three-valued logic by generalizing the matrix method for defining truth functions to three values, a method that Peirce himself originated \citep{anellis_peirces_2012}. Now, Peirce's work on three-valued logic is nowhere near as sustained and complete Łukasiewicz's or Post's. It only spans 6 handwritten pages in Peirce's logic notebook. Nonetheless, when we observe Peirce wrestling with how to handle an additional truth value, this is striking and speaks to what we might call his logical instincts. It seems a natural question to ask why he saw fit to do this. What concerns had Peirce hoped to address with his three-valued logic (which he terms `triadic logic')? This is the question I hope to address throughout the course of this thesis.

The reasons others have taken up three-valued logics are quite diverse. Some, Łukasiewicz most notably, have been motivated by worries about future contingent propositions.  Some use an additional truth value in an attempt to deal with vague predicates and the Sorites Paradox. Others thought that results in quantum mechanics necessitated a third value. And others still have wanted to accommodate undecideable statements in mathematics. Generally speaking, in mainstream treatment of three-valued logic, the third value is meant to account for either possibility, accounting propositions that could turn out either to be true or false, or a kind of borderline indeterminacy, where propositions are really neither true or false. The fact that Peirce writes so little about what he was trying to capture makes the question of his motivations an interesting one.

The strongest indication of what is motivating Peirce to conduct his three-valued experiments comes from the last of the connected pages in his notebook. There he gives two indications that at a glance appear to take us in entirely distinct directions. The chief aim of this thesis is to rebut this view and demonstrate that there is an intimate connection between these two lines of thought.

The first indication is this statement characterizing his triadic logic: ``Triadic Logic is that logic which, though not rejecting entirely the Principle of Excluded Middle, nevertheless recognizes that every proposition, $S$ is $P$, is either true or false, \textit{or else S has a lower mode of being such that it can neither be determinately $P$, nor determinately not $P$, but is at the limit between $P$ and not $P$}'' (MS 339)\footnote{Throught this document I will often refer to passages written by Peirce in published and unpublished collections. It will be convenient to use the typical abbreviated citations for these. Passages from \textit{The Collected Papers of Charles S. Peirce} will be referred to in the text by CP v.p, where v is the volume and p is the paragraph number. Citations from \textit{New Elements of Mathematics} will be abbreviated to NEM v:p, where p will be the page number. \textit{The Essential Peirce} by EP x:y where x is the volume and y is the entry number. Unpublished manuscripts will be referred to as MS followed by the manuscript number. Most references to unpublished manuscripts will be to MS 339, Peirce's logic notebook. I refer to these pages by seq.xyz, according to the order in which they appear in the Harvard Mirador reproduction of this manuscript.} (My emphasis). Peirce's modes of being are another symptom of what has been called his triadism: his insistence in making tripartite divisions in virtually every aspect of philosophy. In his view there are precisely three modes of being, and these have a strictly modal interpretation. The first is possibility, second is actuality, and third is roughly necessity. He believed these modes of being applied to ideas as well as things in nature. Because these subjects have a lower mode of being, they are at the limit between the predicate and its denial. These propositions receive the third truth value, $L$, which is to be interpreted as limit, or not \textit{determinately} true nor false.
%Think about ditching explanation of modes of being. Maybe just use future contingents 
%Peirce's modes of being are another symptom of what has been called his triadism: his insistence in making tripartite divisions in virtually every aspect of philosophy. In his view their are precisely three modes of being, and these have a strictly modal interpretation. The first is possibility, second is actuality, and third is roughly necessity. He believed these modes of being applied to ideas as well as things in nature.

The second indication comes from a rather odd example on the same page: ``Thus, a blot is made on a sheet. Then every point of the sheet is unblackened or blackened. But there are points on the boundary line; and these points are incapable of being unblackened or of being blackened, since these predicates refer to the area about $S$ and a line has no area about any point of it'' (MS 339). Reserving discussion of the oddities of this example for later, it will become clear in subsequent chapters that it is connected to Peirce's views on continuity, continua, and breaches of continuity.

Peirce's understanding of continuity is a somewhat vexed topic. Throughout his life, and especially in the last 25 years, he constantly revised and re-evaluated his definition of it. In the mathematical sense, he came to understand continuity in about the same way as Cantor or Dedekind. However, Peirce's interest in continuity went far beyond mathematics. It extended to time, modality, and plays a central role in his hypothetical cosmology. All of this stems from a doctrine he endorsed and called synechism, which he held some version of going back as far as 1868 (CP 6.103, see also EP 1:2-4). Synechism, in Peirce's words, is ``the tendency to regard everything as continuous'' (EP 2:1). We will briefly see the genesis of his definition of continuity in Chapter 3, Section 2, and see how time, modality and cosmology fit into it in Chapter 4.

So far there have been two attempts to explain Peirce's philosophical motivations behind triadic logic; the first is due to Max Fisch and Atwell Turquette \citeyearpar{fisch_peirces_1966}, the second is due to Robert Lane \citeyearpar{lane_peirces_1999}. Each of these accounts takes a different one of the just mentioned clues as a starting point and arrive at seemingly distinct conclusions. Fisch and Turquette link Peirce's comment about modes of being to his special theory of modality, called `triadic modality.' Lane, on the other hand, denies that triadic modality figures into Peirce's motivations, and instead tries to locate the project entirely within his views on continuity.

Fisch, Turquette, and Lane have focused only on the three pages in Peirce's logic notebook in which he explicitly uses a third truth value. However, there are additional pages that are clearly connected to his triadic logic that can be included in this discussion. Since a digital version of the notebook has now been published \href{https://iiif.lib.harvard.edu/manifests/view/drs:15255301$637i}{online}, it is now much easier to examine these pages and present evidence that better allows us to determine Peirce's motivations. In light of this evidence, I will argue that the two seemingly opposed views on this topic are not actually incompatible and present a synthesis that brings us much closer to Peirce's true motivations.

%Be briefer and don't go into specific 

In the next two chapters I will survey and critically review nearly\footnote{Some of Turquette's papers have been cited but not discussed. Many of these papers are dedicated to purely formal results that can be generated from the page I refer to as seq.640. While these results are not uninteresting, it is highly unlikely that Peirce would have been aware of them and so are unlikely to help advance our understanding of what triadic logic was intended for in the first place. However, I have included discussion of Turquette's findings which Peirce might feasibly have noticed or known about.} all of  what has so far been written about Peirce's triadic logic. I will begin with what Peirce himself wrote in connection with this logic and then turn to more recent work on the topic by Turquette, Fisch, and Lane.

In Chapter 2 I give as close to an exhaustive presentation of Peirce's experiments in his notebook as possible. The first three pages I discuss have been taken up by the commentators just mentioned. The first of these is usually thought to mostly be a failed experiment, although it did yield some useful one-place operators. The second more successfully defines a set of three-valued connectives, all versions of conjunctions and disjunctions. The final page contains Peirce's most explicit statement of what his triadic logic was intended for and seems to reference some of his comments on both triadic modality, as well as continuity. Out of these three pages, the first two are \textit{versos} and only the third is a \textit{recto}.

%In the second half of chapter two, I discuss pages that are connected with triadic logic, but that the above commentators have left out of there discussion. The first of these contains examples of Peirce's existential graphs, his preferred representation of logic, but with an additional operator that does not appear to be used elsewhere. Peirce seems pretty quickly to have decided his graphical representation was not up to the job, as he reverted back to his symbolic notation on all subsequent pages. The second of these unmentioned pages makes reference to ``special universes'' as well as the ``fundamental quadratic'' of Boolean logic. The third of these seems to be mostly scratchwork with a concluding remark about functions on values being ``known'' and ``generally known.'' All three of these pages are \textit{rectos}, the first two of which are on the front side of the \textit{versos} in the first half of the section.

In chapter 3, I discuss Fisch and Turquette's paper, ``Peirce's Triadic Logic,'' and Lane's response, ``Peirce's Triadic Logic Revisited.'' These are the most well known accounts that attempt to describe the philosophical issues that may have motivated Peirce to experiment with three-valued logic. Fisch and Turquette are right that the notebook pages make a compelling case that triadic logic is connected with notions of modality and Peirce's tychism. After reviewing this, I turn to Lane's alternative account of Peirce's motivations. While I agree with Lane that continuity is an important piece of the puzzle, I argue against his claim that modality has nothing to do with it. Removing modality from our consideration entirely would involve ignoring an important part of the little historical evidence we have for why Peirce invented triadic logic. I conclude the section by providing evidence that demonstrates that the claim that Peirce was motivated by modality and the claim that he was motivated by continuity are in fact compatible. This view meshes well with much of Peirce's  other discussion of continuity and continua.

In the final chapter, I work out this view and show that we cannot properly understand the purpose of Peirce's triadic logic exclusively through modality or continuity alone. In the first section, I explain why Peirce thought his subject matter required him to deviate from classical logic. In the second, I answer the question as to why he needed his additional truth value, $L$. In the third, I give examples of the kinds of propositions that would be evaluated as $L$. I admit that this unification is by no means seamless. However, this is not necessarily a deficiency for this account but rather a natural result of its subject matter. Peirce's triadic logic was based on ideas that he himself never fully worked out to his own satisfaction. He continued writing about the ideas expressed here until he died in 1914 and he was proud to admit his own uncertainty.